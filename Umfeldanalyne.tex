%Beginn Dokumentdefintion und Paketimportierung
\documentclass[11pt,a4paper]{article}
\usepackage{ngerman}
\usepackage[utf8]{inputenc}
\begin{document}

%Beginn Überschrift
\begin{center}
{\huge \bf PORG - Umfeldanalyse Informatik} \\
\end{center}
\bigskip
%Beginn eigentlicher Text
{\large \bf Ähnliche Projekte}
\smallskip \\
Von der Informatik gesehen ist das Projekt eine Kombination aus bereits vorhandenen Teilbereichen:
\begin{itemize}
\item MMORPG (Netzwerkkommunikation/Accountsystem)
\item 3d-Charaktersteuerung/Grafik und Physik
\item Go-Spielserver
\item Go-KI
\end{itemize}
Entsprechend kann man sich einiges aus vorhandenen Projekten nehmen:
\begin{itemize}
\item Bekannten Open-Source-Rollenspielen wie PlaneShift
\item Unreal-Engine
\item OGS, kaya.gs
\item Gnugo und viele andere
\end{itemize}
Entsprechend sind dies auch die Teilprodukte von uns, die für andere interressant sind.
\bigskip \\
{\large \bf Unterstützungsmöglichkeiten}
\smallskip \\
Die Informatik produziert Code, deswegen wird hauptsächlich Manpower benötigt. Gute Möglichkeiten diese zu erhalten:
\begin{itemize}
\item Informatik- Studenten und Auszubildende die Erfahrung mit dem gelernten Sammeln wollen. Eventuell Uniprojekte.
\item Gospielende Informatiker
\end{itemize}
Das einzige Sonstige für die entscheidende für die Informatik sind Testserver, die über Sponsorengelder finanziert werden können.
\bigskip \\
{\large \bf Zielgruppe}
\smallskip \\
Das Ziel ist es eine Umgebung zu bieten, die das Lernen von Go unterstützt und Interaktion der Lernenden fördert. Dazu sind folgende Dinge zu beachten:
\begin{itemize}
\item Einfachkeit der Installation (Standalone Client vs. integrierter Client)
\item Möglichkeiten Spielinhalte außerhalb des eigentlichen Clienten anzubieten
\item Intuitives, aber mächtiges Chatsystem mit Funktionen wie Freundesliste, Emotes, verschiedenen Chatebenen/Räumen
\item Überwachung der Kommunikation auf unerwünschte Dinge mit Modsystem, Chatfilter
\end{itemize}
Diese Dinge sind mit der Leistung des Spiels und der Beanspruchung von Ressourcen sowohl auf dem Server als auch dem Computer des Spielers abzuwägen.
\bigskip \\
{\large \bf Zu beachtende Dinge}
\smallskip \\
Es gibt keine momentan absehbaren Gruppen die der Informatik kritisch gegenüberstehen. Trdtzdem gibt es einige Regeln, damit dies so bleibt:
\begin{itemize}
\item Bei Inspiration von anderen Projekten die Lizenz beachten
\item Gute Kommentierung um potentielle Weiterentwickler nicht abzuschrecken.
\end{itemize}
Das einziege sonstige für die entscheidende für die Informatik sind Testserver, die über Sponsorengelder finanziert werden können.

\end{document}